\section{Introduction}  

In this section we will briefly introduce the aim of this project and the algorithm that we studied and implemented in a parallel fashion.
The algorithm is the Gauss-Newton algorithm for non linear least square interpolation.
In this particular case we studied an implementation of the algorithm for three dimensional interpolation of images representing gaussian beams.
The sequential version of the algorithm was developed in summer 2009 for interpolation of images of laser beams monitored by digital camera. 
By interpolation it is possible to monitor the ouput of a high definition camera with a precision down to micrometers, this operation requires lots of computation nonetheless.
This constitute the main reason for a parallel version of the program.

\subsection{Experimental Setup}

In this section we will briefly illunstrate the enviroment where the algorithm is used and the main purpose for its developement.
Laser beams are used to monitor the position and orientation of suspended mirrors used in the interferometer. 
The beam is pointed towards the mirror with a 45 degree angle and reflected on the screen of a high resolution digital camera.
Tilting of the mirror will produce a movement in the position of the laser beam while distortion of the surface of the mirror will produce a change in focus (size) of the beam.
Using then multiple cameras is possible to have a complete three dimensional displacement of the mirror in real time.
In order to do so position and size of the beam should be monitored with high precision. 
As said before the best way to do so is to interpolate the profile of the captured image of the laser beam which, in normal operative conditions, have a perfect gaussian shape.
Since the output of the camera is used for adjustments on remote controlled machinery it is important to have both a high throughput in terms of number of images elaborated and high precision of the interpolation itself.
In our opinion this was a nice real life case for studying parallelization.
Now let's analyze the algorithm used for interpolation.

\subsection{Algorithm}

The algorithm is the well known Gauss-Newton algorithm for non linear least square problems.
As the name says this algorithm does not perform perfect interpolation but tries to reduce as most the square error.
It is also applied for a non linear function (gaussian).
Linear methods as polinomial interpolation can be solved in a single step operating on every point (pixel) only once, non linear methods are iterative and so tends to converge after a  number of steps.
The pseudocode of the algorithm is shown in Figure \ref{code:gauss}.

\begin{figure}[h]
\begin{center} 
\begin{minipage}[c]{.85\textwidth} 
\centering 
\begin{pseudo}{}{gauss}
unsigned char $image[num\_pixel]$;
double $parameters$[8], $delta\_vector[8]$, $diff\_vector[num\_pixel]$, $gradient\_matrix[num\_pixel][8]$, $gradient\_result[8][8]$, $gradient\_vector[8]$;
double $error$, $threshold$;

get_image($image$);
get_parameters($parameters$);

while ($error$ > $threshold$ ) {
    for($i$ = 0; $i$ < $num\_pixel$; $i$ ++){
	
          $diff\_vector$[$i$] = $image$[$i$] - evaluateGaussian($parameters$,$i$); 
          $gradient\_matrix[i][:]$ = computeGradients($i$,$parameters$);
	      
    }
    $gradient\_result$ = $gradient\_matrix^T$ * $gradient\_matrix$;
    $gradient\_vector$ = $gradient\_matrix^T$ * $diff\_vector$;
    $delta\_vector[:]$ = LU_solve ($gradient\_result$ , $gradient\_vector$);
    $parameters[:]$ += $delta\_vector[:]$;
    $error$ = compute_square_error($image$, $parameters$);
}
return $parameters$;

\end{pseudo}
\end{minipage} 
\end{center} 
\caption{Pseudocode of Gauss-Newton algorithm}
\end{figure}

It is important to point out some interesting facts.
Ideally the algorithm could loop for an infinite number of iteration but since we are dealing with finite arithmetics the algorithm would surely converge after a finite number of steps.
Consider that since we are monitoring a slowly changing enviroment, one iteration is sufficient in order to converge to a result.
We also had to develop a simulator for the digital camera.

\subsection{Parallel Schemes}

As a first thing we have to model the parallel computation using a structured parallel programming scheme. We know that there is an (ideal) infinite stream of images, hence a way to parallelize this program can be the farm approach: we can replicate the algorithm in each worker following the cost model that we will introduce soon. For this scheme is not necessary to know the internal structure of the algorithm, but if we analyze it, we recognize some interesting aspects.

Following the pseudocode in Figure 1 we have:
\begin{enumerate}
\item the construction of the \textit{diff\_vector} and the \textit{gradient\_matrix} starting from the image.
\item the multiplication between the \textit{gradient\_matrix} and its own transposed matrix, that is the main operation.
\item the multiplication between the transposed matrix and the \textit{diff\_vector}.
\item the LU\_solve operation
\item the sum of the vector \textit{parameter} with \textit{delta\_vector}.
\end{enumerate}

From above, we can notice that the first three points could be executed in local in a module if we know a partition of the \textit{gradient\_matrix} and a partition of the \textit{diff\_vector}. In particular, if we consider very fine grain computations, we notice that it is sufficient to have a single element of the matrix and a single element of the vector for computing the partial multiplications between elements. This means that there is no need of communications of values between different partitions of the matrix but it is necessary to communicate the partial results in such way it is possible to sum them and to obtain the $8$x$8\ gradient\_matrix$ and the vector $gradient\_vector$. Finally, we notice that the fourth and the last point must be executed only after the calculations above and they take a negligible time respect than the previous operations.

This situation suggests a parallelization scheme in which the data can be partitionized between different modules for computing the first three points and after we need a reduce communication for obtain the final results and execute the last two parts of the algorithm. This means that our program can be parallelized through a map reduce too.

At this point we chose to realize both a farm and data parallel scheme. We will further discuss their cost model in the next section.