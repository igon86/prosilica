\section{Conclusions}

There are some aspects come out regarding the different choices and the results of the tests that are very interesting and we will discuss them in this section.

The first thing that we can notice from the results of the test \ref{chart:ottavina_farm_1000} is the the fact that, on demand or round robin scheduling, does not affect application performance.
This agrees with the remarks made in Section \ref{farm_cost}.

In the test \ref{chart:ottavina_data_1000} we see that the data parallel scheme can operate in architectures with few nodes with considerably good results.
Instead, this is not possible in the farm solution that we know require at least three nodes. 
For different parallelism degrees of this application, we have a better behaviour of the data parallel scheme. 
This agrees with the remarks made in Section \ref{data_cost} and depends on the memory occupation of the data parallel which is less than the task farm case. 
This effect is noticeable because we are working on a shared memory architecture so data sets of different workers are all stored in the same principal memory.

The test \ref{chart:ottavina_alldata_1000} tells us that the various implementations of data parallel are not particularly different in terms of scalability but they use a number of service processes (zero, one or two processes) that in an architecture with a few nodes can impact the performance.
We also notice that our implementation of scatter and reduce communications have similar performances with respect to \textit{MPI} primitives.
This is mainly due to the really low impact of communication on this architecture as we will see on \ref{graf:Communication}.

Test \ref{chart:ottavina_camera} just shows what happens introducing the external camera module. 
As we can see performance are only slightly affected.

Test \ref{chart:axt} is performed on a distributed memory platform. 
In this case scalability is far from optimal, farm implementation shows an abrupt stop while data parallel have some kind of graceful degradation.
This could be because of communication cost so we tested the application on a different distributed memory application.
On pianosa \ref{chart:pianosa} scalability is a lot better although general performance is quite worse.
Pianosa exploit less powerful CPUs with respect to the AXTH lab and an almost identical inter connection network.
This enforce the hypothesis that performance are limited by communication costs.

On graph \ref{graph:communication} we can see how communications are affecting the performances.
On AXTH laboratory, communication cost is limiting performance around 100ms.
On pianosa, even if communication cost is higher, scalability is good since computation time is proportionally higher.
We can also notice how in distributed platform and with limited data sets the difference between farm and data parallel implementation is negligible.

Graphs \ref{graf:ottavina_datascale}, \ref{graf:pianosa_datascale} \ref{graf:axth_datascale} just show the behaviour of the application on previous architectures varying the data set (dimension of the images). 
This is useful to have an idea of general behaviour of the application on different architectures.

We can see that for reasonable size of the data (around 1M-10M) similar service time can be obtained by both axth lab and ottavina around 100ms which in turn gives a frame rate of around 10 frames/s.
We can also conclude that for the real case, distributed platform are excessive due to the limited size of the data set and multicore platforms should be used.