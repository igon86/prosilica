\section{Conclusions}

There are some aspects come out regarding the different choices and the results of the tests that are very interesting and we will discuss them in this section.

The first thing that we can notice from the results of the test \ref{chart:ottavina_farm_1000} is the the fact that, on demand or round robin scheduling, does not affect application performance.
This agrees with the remarks made in Section \ref{farm_cost}.

In the test \ref{chart:ottavina_data_1000} we see that the data parallel scheme can operate in architectures with few nodes with considerably good results.
Instead, this is not possible in the farm solution that we know require at least three nodes. 
For different parallelism degrees of this application, we have a better behaviour of the data parallel scheme. 
This agrees with the remarks made in Section \ref{data_cost} and depends on the memory occupation of the data parallel which is less than the task farm case. 
This effect is noticeable because we are working on a shared memory architecture so data sets of different workers are all stored in the same principal memory.

The test \ref{chart:ottavina_alldata_1000} tells us that the various implementations of data parallel are not particularly different in terms of scalability but they use a number of service processes (zero, one or two processes) that in an architecture with a few nodes can impact the performance.
We also notice that our implementation of scatter and reduce communications have similar performances with respect to \textit{MPI} primitives.
This is mainly due to the really low impact of communication on this architecture as we will see on \ref{graf:Communication}.

Test \ref{chart:ottavina_camera} just shows what happens introducing the external camera module. 
As we can see performance are only slightly affected.



